\documentclass[12pt,a4paper,final]{report} % When using the draft no
                                           % pictures are included in
                                           % the document
\usepackage[utf8]{inputenc}
\usepackage[UKenglish]{babel}

% ==========================================================================
% IMPORTANT: To install latex run
%
% sudo apt-get install texlive-latex-base
%
% When some packages are not found try (recommended to install after
% latex-base)
%
% sudo apt-get install texlive-latex-extra
%
% If you are using Biber as the bibliography manager, then you need to
% ensure you have the file biblatex.sty installed, if you are not sure
% run the following command to install it
%
% sudo apt-get install texlive-bibtex-extra
%
% You will also need to install Biber from the "Ubuntu Software
% Centre", and if you want to use it with "Texmaker" you need to
% configure it to use biber insteat of bibtex at Options > Configure
% Texmaker
%
% ==========================================================================

\usepackage{amsmath}
% BEGIN ---- The block of code after this is to write augmented
% matrices and allows you to align entries in matrices
\makeatletter
\renewcommand*\env@matrix[1][*\c@MaxMatrixCols c]{%
  \hskip -\arraycolsep
  \let\@ifnextchar\new@ifnextchar
  \array{#1}}
\makeatother

% Example
% -------------------------------------------------------------------------
% \begin{equation}
%   \begin{bmatrix}[rrr]
%     2 & -1\\
%     -1 & 2
%   \end{bmatrix}
%   \begin{bmatrix}
%   x\\
%   y
%   \end{bmatrix}
%   =
%   \begin{bmatrix}
%   0\\
%   3
%   \end{bmatrix},
% \end{equation}
% ------------------------------------------------------------------------

% END ---- The block of code before this is to write augmented
% matrices and allows you to align entries in matrices

\usepackage{amsfonts}
\usepackage{amssymb}

% To work with figures, inserting and adding a relative directory to
% search for figures
\usepackage{graphicx}
\graphicspath{ {figures/} } % There must not be space before and after
                            % "figures/" for the second pair of "{"
                            % "}". See
                            % https://www.sharelatex.com/learn/Inserting_Images

\usepackage{caption} % Extend the formats for captions
\usepackage{subcaption}

\usepackage{url} % To add url-format to the document
\usepackage{mathtools} % More tools than those offered by "amsmath"

% ========================================================================
% Algorithm packages ----------------------------- BEGIN -----------------
% ========================================================================
%\usepackage[ruled]{algorithm} % Support for algorithms
%\usepackage{algpseudocode}
% ========================================================================
% Algorithm packages ----------------------------- END -------------------
% ========================================================================

% ========================================================================
% Bibliography stuff ----------------------------- BEGIN -----------------
% ========================================================================

% At the end of the document we need to call \printbibliography to
% include the bibliography. Use this command whenever you find that
% the biblatex.sty package is not installed
%
% sudo apt-get install texlive-bibtex-extra
%
% We need to install Biber from the "Ubuntu Software Centre", and if
% you want to use it with "Texmaker" you need to configure it to use
% biber insteat of bibtex at Options > Configure Texmaker

\usepackage[backend=biber,style=ieee,sorting=nyt]{biblatex}
% Imports the package to manage the bibliography (the option backend
% set biber as the backend-- which I dont know what it is ---, the
% style is how the bibliography is cited and printed at the
% bibliography section, and the option sorting=nyt sorts the
% bibliography by name=n, year=y and title=t).
\addbibresource{code_style.bib} % Imports the bibliography file

\usepackage{csquotes} % Add this, otherwise a warning is thrown if not
                      % included

% ========================================================================
% Bibliography stuff ----------------------------- END -------------------
% ========================================================================

% The listing package to add formated blocks of code
\usepackage{listings}

% MY definitions (short-cuts)
% For vectors
\newcommand{\vect}[1]{\mathbf{#1}}

\author{tachidok}
\title{Report}

\begin{document}

% Enable the following when using a report docuemtn style

%\tableofcontents
%\listoffigures
%\listoftables

\chapter{Coding style}
Here we present a kind of coding/programming style rules to happily
coding together. First of all, as you may already noticed, we use C++
as our programming language. Why? well, that is because of its widely
continuous spreading along the scientific community and all its nicely
offered features.

The list of rules that you'll find here were taken from our own
experience. We took what has worked for us when working in cooperative
projects, and set aside those silly things that only made cooperation
painful and nol longer fun. Hope you agree with us in the list of
rules we present here, thus, sit back, relax and happy reading.

\section{Comments}

``\textit{Comments are supposed to make your code easier to understand
  and maintain-- not harder}'', \cite{when_comments_go_bad:URL}.

There is nothing more frustrating than having to review someone's else
code (and sometimes our own code) and find out that the variables or
functions names are not representative of what they are used for, the
code is a complete mess having no identation at all, and the worse
thing is that there is not a single commented line to give us a clue
of what is going on. In front of this situation we have two options,
\begin{itemize}
\item spend a lot of time and effort to understand the code (and find
  out that the code is not doing what you are looking for), or
\item throw it away and do your own implementation (I personally
  prefer this option).
\end{itemize}

Certaintly, the above prevents code reusing which is one of the main
purposes of this library.

First of all, try to make your code as self-explanatory as
possible. Suppose that you are telling a story to a group of people
and you want everyone that is hearing you do not get lost while you
are talking. If the syntaxis of the programming language does not
allow you to write your story clearly then use comments, do not be
afraid of them.

Put your comments as close as possible to the source code they are
referring. A developer may not notice a comment that referrs to a code
line that he/she is modifying, thus the comments and the code will be
out of sync, \cite{when_comments_go_bad:URL}.

Use comments to clarify the selection of some algorithms or to mention
things to rememeber during the execution of the code.

When writing functions that perform a large number of sub-tasks, use
the initial part of it to write a (brief!) summary of the followed
strategy by the function and include a list with the main steps of the
function. This way the reader of your code gets an idea of what to
expect in the body of the function.

One final thing, always keep in mind that what may be transparently
obvious for you, may be completely obscure and complex for another
developer (or for yourself after returning from a nonprogramming
period), use comments to ease the reading of your code, not to make it
harder.

If you are looking for some good reasons for not documenting your code
then check see these ones
\cite{why_programmers_do_not_comment_their_code:URL} and see what
better matches with you.

\section{Indentation}
This is simple, indent your code for easy reading. I use emacs as my
preferred editor and have it configured to automatically indent with a
single whitespace when pressing TAB. You can copy and use my emacs
configuration file (.emacs) from the the \textbf{tools} folder or
configure your favorite editor to indent with a single whitespace.

\section{Variables}
We use variables to store values or data that are used frequently in
the body of a function or as part of a class.

\subsection{Variables names}
Use variable names that reflect the values or data stored in the
variable. If you require to store the velocity you should use
\texttt{v} or \texttt{velocity} as the variable name. 

Avoid using the well known variables names:

\begin{table}
  \centering
  \begin{tabular}{|c|c|c|}
    \hline
    \texttt{var1} & \texttt{var2} & \texttt{varn}\\
    \hline
    \texttt{value} & \texttt{variable} & \texttt{my\_variable}\\
    \hline
    \texttt{this\_variable} & \texttt{other\_variable} &
    \texttt{another\_variable}\\
    \hline
    \texttt{delete\_variable} &  \texttt{here} & \texttt{there}\\
    \hline
\end{tabular}
\label{tbl:bad_variables_names}
\caption{Bad variables names.}
\end{table}
I know you (use?) have used those variable names, everyone does it.

Variables that stores the number of elements of something must use the
prefix {\verb n_ }. For example, if you want to store the number of
processors, the variable must be named as follow:

\begin{lstlisting}[language=C++]
  const unsigned n_processors = nprocessors();
\end{lstlisting}

\subsection{Variables types}
Use \texttt{unsigned} instead of \texttt{int} in loops (\texttt{for}
or \texttt{while}) that do not require negative indexes.

\subsection{\texttt{Const} or variable?}
Well, it happens that in C++, variables that do not pretend to change
their value along the entire exection of the program are declared with
a \texttt{const} before the variable type. They why are they still
called variables, well you may be interested to read this.

Anyway, use \texttt{const} on variables that are not intented to
change their value. Remember that when using \texttt{const} you need
to specify the value of the variable at the time of its declaration.

\section{Functions}
Functions are a great idea that let us split a complicated tasks in
small (or not that small) and easy to digest sub-tasks. We can
implement a complex task as a set of subtasks, each implementing a
basic idea that may be re-used in other complex tasks.

Think of a function as an independent task that may even call other
functions to perform its job. A funtion

\subsection{Functions types}

\subsection{Functions names}
When working in small-individual project it is quite tempting to use
short name functions, first because no one else will use (or review)
our code, and second because of laziness. We pretend that this library
be (re-)used by a large community, thus function's names that reflect
the intention or work performed by the function is a good way to
promote reusability.

\begin{itemize}
  \item Functions names MUST all be in lowercase.
  \item Use ``\_'' to separate words in the function name.
\end{itemize}

\subsection{Split large funtions into sub-task}

\subsection{Input and output arguments}

\subsubsection{Const or non const}
Use ``const'' as much as you can, if you do not need (or do not know)
to change the value of a variable inside a function then use ``const''
before the type of the variable\\

\texttt{const unsigned no\_modified\_variabled} (why is it called a
variable then?)
\\
otherwise do not use ``const'' \texttt{unsigned modified\_variable}.


\subsubsection{Pass by copy or pass by reference}
Only pass arguments by copy when they are a single value, such as an
integer or a double value. Any other argument MUST be passed by
reference. This is to avoid copying large vectors, matrices or objects
and thus run out of memory because of the many copies of the same
object in memory. If we do not really need a copy of every single
element in a vector, matrix or object then why should we make a copy
ot it?

Examples of passing arguments by reference here soon

Use \& when passing an argument by reference

% ========================================================================
% Customising bibliography ------------------- BEGIN ---------------------
% ========================================================================
\printbibliography[heading=bibintoc]
% Print bibliography from articles
%\printbibliography[heading=bibintoc, type=article,title={Articles}]
% Print bibliography from books
%\printbibliography[heading=bibintoc,type=book,title={Books}]
% Print bibliography from incollection
%\printbibliography[heading=bibintoc,type=incollection,title={In collection}]
% ========================================================================
% Customising bibliography ------------------- END -----------------------
% ========================================================================

\end{document}
