% Created 2020-05-12 mar 16:45
\documentclass[11pt]{article}
\usepackage[utf8]{inputenc}
\usepackage[T1]{fontenc}
\usepackage{fixltx2e}
\usepackage{graphicx}
\usepackage{longtable}
\usepackage{float}
\usepackage{wrapfig}
\usepackage{rotating}
\usepackage[normalem]{ulem}
\usepackage{amsmath}
\usepackage{textcomp}
\usepackage{marvosym}
\usepackage{wasysym}
\usepackage{amssymb}
\usepackage{hyperref}
\tolerance=1000
\author{Julio C.}
\date{\today}
\title{Installing SuperLU}
\hypersetup{
  pdfkeywords={},
  pdfsubject={},
  pdfcreator={Emacs 25.2.2 (Org mode 8.2.10)}}
\begin{document}

\maketitle
\tableofcontents

This guide is to allow \href{https://github.com/tachidok/chapchom}{chapchom} to work with \texttt{SuperLU 5.2.0} in
\begin{itemize}
\item \texttt{Ubuntu 16.04 LTS 64 bits},
\item \texttt{Ubuntu 18.04.2 LTS 64 bits}.
\end{itemize}
If you find any problem or something in this guide makes no sense
please contact the author.

Please also refer to the \href{http://crd-legacy.lbl.gov/~xiaoye/SuperLU/}{original documentation} in case you are having
troubles with the installation.

\section{Requirements}
\label{sec-1}
Before installing this version of SuperLU please check that you have
not any other version of SuperLU already installed in your system. If
that is the case then proceed to uninstall it.

Once you ensure that SuperLU is not already installed in your system
then proceed to check that you have the following packages installed
in your system.
\begin{itemize}
\item In \texttt{Ubuntu 16.04 LTS 64 bits}
\begin{itemize}
\item \texttt{cmake 3.5.1-1ubuntu3}
\end{itemize}
\item In \texttt{Ubuntu 18.04.2 LTS 64 bits}
\begin{itemize}
\item \texttt{cmake 3.10.2}
\end{itemize}
\end{itemize}

\section{Installation}
\label{sec-2}

\begin{itemize}
\item Extract the compressed file
\texttt{/external\_src/superLU/superlu\_5.2.0.tar.gz} in a folder of your
preference. We recommend you to extract it out of the chapchom
project folder to avoid adding the files to the git repository. If
you do extract it inside the chapchom project folder please extreme
precautions when adding your files to the git repository.

\item Go to the folder where you extracted the files and type
\end{itemize}

\begin{verbatim}
mkdir build
cd build
cmake .. -DCMAKE_INSTALL_PREFIX=../lib
\end{verbatim}

\begin{verbatim}
mkdir build
cd build
cmake .. -DCMAKE_INSTALL_PREFIX=../lib
\end{verbatim}

the last line indicates where to perform the installation, I use to
install it in the \texttt{lib} directory of the SuperLU folder, that is why
that value. However, if you have root privileges then you may not need
to specify a value for the \texttt{CMAKE\_INSTALL\_PREFIX} variable.

NOTE: If you are installing \texttt{Armadillo} with \texttt{SuperLU} support then
you may need to install \texttt{SuperLU} with the flag \texttt{-fPIC} (which stands
for 'Position Independent Code'), to do so open the \texttt{CMakeLists.txt}
file in the main folder of \texttt{SuperLU} and edit the line where \texttt{CFLAGS}
are added (in \texttt{SuperLU 5.2.0} it is in line \texttt{68}). This is how it
looks for \texttt{version 5.2.0}.

\begin{verbatim}
set(CMAKE_C_FLAGS "-fPIC -DPRNTlevel=0 -DAdd_ ${CMAKE_C_FLAGS}")
\end{verbatim}

\begin{itemize}
\item When the process finish then type
\end{itemize}

\begin{verbatim}
make
\end{verbatim}

You can try \texttt{make -j \# of processors} instead of \texttt{make} to use more
processors at compilation time.

\begin{itemize}
\item Install it \ldots{}
\end{itemize}

\begin{verbatim}
make install
\end{verbatim}

\begin{itemize}
\item And finally, run the test by typing
\end{itemize}

\begin{verbatim}
ctest
\end{verbatim}

You can check the results of the testing process in the following
files

\begin{center}
\begin{tabular}{ll}
\texttt{build/TESTING/s\_test.out} & \texttt{single precision, real}\\
\texttt{build/TESTING/d\_test.out} & \texttt{double precision, real}\\
\texttt{build/TESTING/c\_test.out} & \texttt{single precision, complex}\\
\texttt{build/TESTING/z\_test.out} & \texttt{double precision, complex}\\
\end{tabular}
\end{center}
% Emacs 25.2.2 (Org mode 8.2.10)
\end{document}